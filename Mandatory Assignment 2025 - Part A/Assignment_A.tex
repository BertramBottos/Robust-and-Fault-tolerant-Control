\documentclass[a4,11pt]{article}
\usepackage{rotating}
\usepackage{graphicx,color,psfrag}
\usepackage{longtable}
\usepackage{multirow}
\usepackage{amsmath}
\usepackage{breqn}
\oddsidemargin -13mm
\evensidemargin -13mm
\textwidth 177.6mm
\title{Structural Analysis Toolbox Report}
\author{This document is autogenerated.}
\begin{document}
\maketitle
\section{Overview}
The investigated system has
\begin{itemize}
	\item 15 constraints $\mathcal{C} = [$ $c_1$, $c_2$, $c_3$, $c_4$, $c_5$, $c_6$, $m_{13}$, $m_{14}$, $m_{15}$, $d_7$, $d_8$, $d_9$, $d_{10}$, $d_{11}$, $d_{12}$ $]$,
	\item 5 known variables $\mathcal{K} = [$ $u_1$, $u_2$, $y_1$, $y_2$, $y_3$ $]$ and
	\item 13 unknown variables $\mathcal{X} = [$ $theta_1$, $dtheta_1$, $omega_1$, $domega_1$, $theta_2$, $dtheta_2$, $omega_2$, $domega_2$, $theta_3$, $dtheta_3$, $omega_3$, $domega_3$, $d$ $]$.
\end{itemize}
The constraints of the system are as follows:
\begin{longtable}{ l | p{0.8\textwidth} }
	$c_1$ & $ 0=dtheta_1-omega_1 $ \\
	$c_2$ & $ 0=d-u_1(t)+k_1\cdot \left(theta_1-theta_2\right)+J_1\cdot domega_1+b_1\cdot omega_1 $ \\
	$c_3$ & $ 0=dtheta_2-omega_2 $ \\
	$c_4$ & $ 0=k_2\cdot \left(theta_2-theta_3\right)-k_1\cdot \left(theta_1-theta_2\right)-u_2(t)+J_2\cdot domega_2+b_2\cdot omega_2 $ \\
	$c_5$ & $ 0=dtheta_3-omega_3 $ \\
	$c_6$ & $ 0=J_3\cdot domega_3-k_2\cdot \left(theta_2-theta_3\right)+b_3\cdot omega_3 $ \\
	$m_{13}$ & $ 0=y_1(t)-theta_1 $ \\
	$m_{14}$ & $ 0=y_2(t)-theta_2 $ \\
	$m_{15}$ & $ 0=y_3(t)-theta_3 $ \\
	$d_7$ & $ 0=dtheta_1-\frac{\partial }{\partial t} theta_1 $ \\
	$d_8$ & $ 0=domega_1-\frac{\partial }{\partial t} omega_1 $ \\
	$d_9$ & $ 0=dtheta_2-\frac{\partial }{\partial t} theta_2 $ \\
	$d_{10}$ & $ 0=domega_2-\frac{\partial }{\partial t} omega_2 $ \\
	$d_{11}$ & $ 0=dtheta_3-\frac{\partial }{\partial t} theta_3 $ \\
	$d_{12}$ & $ 0=domega_3-\frac{\partial }{\partial t} omega_3 $ \\
\end{longtable}
The analysis obtained 1 matchings that yield in total 2 parity equations.\newpage
\section{Canonical Decomposition}The system consists of
\begin{itemize}	\item the over-determined subsystem $\mathcal{S}^+$ with $\mathcal{C}^+ = [$ $c_3$, $c_4$, $c_5$, $c_6$, $m_{13}$, $m_{14}$, $m_{15}$, $d_9$, $d_{10}$, $d_{11}$, $d_{12} $ $]$ and $\mathcal{X}^+ = [$ $theta_1$, $theta_2$, $dtheta_2$, $omega_2$, $domega_2$, $theta_3$, $dtheta_3$, $omega_3$, $domega_3 $ $]$,
	\item the just-determined subsystem $\mathcal{S}^0$ with $\mathcal{C}^0 = [$ $c_1$, $c_2$, $d_7$, $d_8 $ $]$ and $\mathcal{X}^+ = [$ $dtheta_1$, $omega_1$, $domega_1$, $d $ $]$ and
	\item the under-determined subsystem $\mathcal{S}^-$ with $\mathcal{C}^- = [$ $ $ $]$ and $\mathcal{X}^+ = [$ $ $ $]$.
\end{itemize}
\section{Incidence Matrix}
Table \ref{tab:matrix} presents the incidence matrix of the investigated system.\setlength\tabcolsep{2mm}

\begin{table}[!htb]
\centering
\normalsize
\begin{tabular}{|l|ccccc|ccccccccccccc|}
\hline
&\multicolumn{5}{c|}{$\mathcal{K}$} & \multicolumn{13}{|c|}{$\mathcal{X}$}\\
\cline{2-19}
\# & \begin{sideways}$u_1$\end{sideways}& \begin{sideways}$u_2$\end{sideways}& \begin{sideways}$y_1$\end{sideways}& \begin{sideways}$y_2$\end{sideways}& \begin{sideways}$y_3$\end{sideways}& \begin{sideways}$theta_1$\end{sideways}& \begin{sideways}$dtheta_1$\end{sideways}& \begin{sideways}$omega_1$\end{sideways}& \begin{sideways}$domega_1$\end{sideways}& \begin{sideways}$theta_2$\end{sideways}& \begin{sideways}$dtheta_2$\end{sideways}& \begin{sideways}$omega_2$\end{sideways}& \begin{sideways}$domega_2$\end{sideways}& \begin{sideways}$theta_3$\end{sideways}& \begin{sideways}$dtheta_3$\end{sideways}& \begin{sideways}$omega_3$\end{sideways}& \begin{sideways}$domega_3$\end{sideways}& \begin{sideways}$d$\end{sideways} \\ 
\hline
$c_1$  & 0 & 0 & 0 & 0 & 0 & 0 & 1 & 1 & 0 & 0 & 0 & 0 & 0 & 0 & 0 & 0 & 0 & 0 \\ 
$c_2$  & 1 & 0 & 0 & 0 & 0 & 1 & 0 & 1 & 1 & 1 & 0 & 0 & 0 & 0 & 0 & 0 & 0 & 1 \\ 
$c_3$  & 0 & 0 & 0 & 0 & 0 & 0 & 0 & 0 & 0 & 0 & 1 & 1 & 0 & 0 & 0 & 0 & 0 & 0 \\ 
$c_4$  & 0 & 1 & 0 & 0 & 0 & 1 & 0 & 0 & 0 & 1 & 0 & 1 & 1 & 1 & 0 & 0 & 0 & 0 \\ 
$c_5$  & 0 & 0 & 0 & 0 & 0 & 0 & 0 & 0 & 0 & 0 & 0 & 0 & 0 & 0 & 1 & 1 & 0 & 0 \\ 
$c_6$  & 0 & 0 & 0 & 0 & 0 & 0 & 0 & 0 & 0 & 1 & 0 & 0 & 0 & 1 & 0 & 1 & 1 & 0 \\ 
$m_{13}$  & 0 & 0 & 1 & 0 & 0 & 1 & 0 & 0 & 0 & 0 & 0 & 0 & 0 & 0 & 0 & 0 & 0 & 0 \\ 
$m_{14}$  & 0 & 0 & 0 & 1 & 0 & 0 & 0 & 0 & 0 & 1 & 0 & 0 & 0 & 0 & 0 & 0 & 0 & 0 \\ 
$m_{15}$  & 0 & 0 & 0 & 0 & 1 & 0 & 0 & 0 & 0 & 0 & 0 & 0 & 0 & 1 & 0 & 0 & 0 & 0 \\ 
\hline
$d_7$  & 0 & 0 & 0 & 0 & 0 & X & 1 & 0 & 0 & 0 & 0 & 0 & 0 & 0 & 0 & 0 & 0 & 0 \\ 
$d_8$  & 0 & 0 & 0 & 0 & 0 & 0 & 0 & X & 1 & 0 & 0 & 0 & 0 & 0 & 0 & 0 & 0 & 0 \\ 
$d_9$  & 0 & 0 & 0 & 0 & 0 & 0 & 0 & 0 & 0 & X & 1 & 0 & 0 & 0 & 0 & 0 & 0 & 0 \\ 
$d_{10}$  & 0 & 0 & 0 & 0 & 0 & 0 & 0 & 0 & 0 & 0 & 0 & X & 1 & 0 & 0 & 0 & 0 & 0 \\ 
$d_{11}$  & 0 & 0 & 0 & 0 & 0 & 0 & 0 & 0 & 0 & 0 & 0 & 0 & 0 & X & 1 & 0 & 0 & 0 \\ 
$d_{12}$  & 0 & 0 & 0 & 0 & 0 & 0 & 0 & 0 & 0 & 0 & 0 & 0 & 0 & 0 & 0 & X & 1 & 0 \\ \hline
\end{tabular}
\caption{Incidence matrix of the investigated system.}
\label{tab:matrix}
\end{table}
\newpage
\section{Matchings}
Table \ref{tab:matchings} lists the obtained matchings. The fields either contain the matched unknown variables, zeros to indicate an unmatched constraints or nothing if constraints are not used in a matching.\setlength\tabcolsep{2mm}

\begin{table}[!htb]
\centering
\normalsize
\begin{tabular}{|c|c|c|c|c|c|c|c|c|c|c|c|c|c|c|c|}
\hline
~ & \textbf{$c_1$} & \textbf{$c_2$} & \textbf{$c_3$} & \textbf{$c_4$} & \textbf{$c_5$} & \textbf{$c_6$} & \textbf{$m_{13}$} & \textbf{$m_{14}$} & \textbf{$m_{15}$} & \textbf{$d_7$} & \textbf{$d_8$} & \textbf{$d_9$} & \textbf{$d_{10}$} & \textbf{$d_{11}$} & \textbf{$d_{12}$}\\ \hline 
\textbf{1} & $omega_1$ & $d$ & $omega_2$ & $domega_2$ & $omega_3$ & $domega_3$ & $theta_1$ & $theta_2$ & $theta_3$ & $dtheta_1$ & $domega_1$ & $dtheta_2$ & 0 & $dtheta_3$ & 0\\ \hline 
\end{tabular}
\caption{Matchings of the investigated system.}
\label{tab:matchings}
\end{table}

\section{Detectability and isolability analysis}
Table \ref{tab:iso} lists the detectability and isolability properties of the parity equations separately and over all combined. Detectable (\textbf{$d$}), isolable (\textbf{$i$}) and non-failable constraints (\textbf{$n$}) are marked accordingly.\setlength\tabcolsep{2mm}

\begin{table}[!htb]
\centering
\normalsize
\begin{tabular}{|c|c|c|c|c|c|c|c|c|c|c|c|c|c|c|c|}
\hline
~ & \textbf{$c_1$} & \textbf{$c_2$} & \textbf{$c_3$} & \textbf{$c_4$} & \textbf{$c_5$} & \textbf{$c_6$} & \textbf{$m_{13}$} & \textbf{$m_{14}$} & \textbf{$m_{15}$} & \textbf{$d_7$} & \textbf{$d_8$} & \textbf{$d_9$} & \textbf{$d_{10}$} & \textbf{$d_{11}$} & \textbf{$d_{12}$}\\ \hline 
\textbf{1} &  &  & $d$ & $d$ & $d$ & $d$ & $d$ & $d$ & $d$ & $n$ & $n$ & $n$ & $n$ & $n$ & $n$\\ \hline 
\textbf{ALL} &  &  & $d$ & $d$ & $d$ & $d$ & $d$ & $d$ & $d$ & $n$ & $n$ & $n$ & $n$ & $n$ & $n$\\ \hline 
\end{tabular}
\caption{Detectability and isolability of the investigated system.}
\label{tab:iso}
\end{table}


\end{document}